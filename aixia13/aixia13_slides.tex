%Andrea Borghesi
%Università degli studi di Bologna

% CP2013 presentation
\documentclass{beamer}
\usetheme{Dresden}


\usecolortheme[rgb={0,0.5,0}]{structure}

\usepackage{indentfirst}
\usepackage[utf8]{inputenc}
\usepackage[T1]{fontenc}
\usepackage{fancyhdr}
\usepackage{graphicx, subfigure}
\usepackage[font={small}]{caption}
\usepackage{eurosym}
\usepackage{hyphenat}


%cartelle contenenti le immagini
\graphicspath{{/media/sda4/missioni/aixia13/immagini/}}

\newcommand{\clpr}{CLP({\ensuremath{\cal R}})}

\title{Multi-agent simulator of incentive influence on PV adoption}
\author[]{\underline{Andrea Borghesi} \and  Michela Milano}

\institute[]{DISI, University of Bologna, Italy}
\date{}

\setbeamercolor{itemize subitem}{fg=orange} % cambia colore elementi sotto-liste
\setbeamercolor{block body}{bg=yellow!20,fg=black}
%\captionsetup[subfigure]{font=small,labelfont=small}


\begin{document}

	\begin{frame}[plain]
		\titlepage
	\end{frame}
	

\section*{Outline}
\begin{frame}
	\tableofcontents
\end{frame}


\section{Introduction}

	\begin{frame}
	\frametitle{ePolicy Project}
		\begin{columns}
			\column{.5\textwidth}
			\begin{itemize}
				\item FP7 STREP Project funded under ICT tools for Governance and Policy Modeling.
				\item AIM: provide decision support systems for policy makers.
				\item Case study: Emilia-Romagna regional Energy Plan.
			\end{itemize}
			\column{.5\textwidth}
			\begin{figure}[hbt]
				\centering
				\includegraphics[scale=0.3]{italyMap}
				\label{italyMap}
			\end{figure}
		\end{columns}
 	\end{frame}
 	
 	\begin{frame}
	\frametitle{Policy Making Process}
		\begin{itemize}
			\item Public policy issues are extremely complex, occur in rapidly changing environments and involve conflicts among different interests.
			\item Policy making in the energy sector accounts electric and thermal energy production, energy efficiency, transports.			
			\item Energy policies strongly affect economic development sustainability and social acceptance.	
 			\item Increasing attention given to sustainable energy policies.
		\end{itemize}
 	\end{frame}
 	
 	\begin{frame}
	\frametitle{Energy Policies}
		\begin{itemize}
			\item EU 20-20-20 initiative
			\begin{itemize}
				\item 20\% reduction of $CO_2$ emissions (w.r.t. 1990 levels)
				\item 20\% of energy produced by renewable sources
				\item 20\% increase of energy efficiency (w.r.t. 1990 levels) 
			\end{itemize}
			\item It should be perceived, and thus enforced, by	national and regional energy policies.
		\end{itemize}
 	\end{frame}
 	
 	
\section{Policy making process}
	\begin{frame}
	\frametitle{Policy Making Cycle}
		\begin{columns}
			\column{.5\textwidth}
			\begin{itemize}
				\item Four steps in the policy making process: 
				\begin{itemize}
					\item Traditionally performed in sequence
					\item An integrated ICT tool for supporting the overall process is missing   
				\end{itemize}
				\item Focus on Planning and Implementation
			\end{itemize}
			\column{.5\textwidth}
			\begin{figure}[hbt]
				\centering
				\includegraphics[scale=0.35]{policyLifeCycle}
				\label{epolicyLifeCycle}
			\end{figure}
		\end{columns}
 	\end{frame}
 	
\subsection{Planning phase}
	\begin{frame}
	\frametitle{Planning phase}
			\begin{itemize}
			\item In the planning step, strategic objectives are set, budget constraints are defined, geophysical constraints are considered.
			\item The policy makers should define the amount of planned activities.
			\begin{itemize}
				\item Primary activities: energy plants (PV, biomasses, fossil fuels plants..)
				\item Secondary activities: infrastructures (roads, power lines..)
			\end{itemize}
			\item Both type of activities have impacts on the environment and costs.
			\item Building a plan means solving an optimization problem, satisfying constraints on impacts, costs, etc.
		\end{itemize}
	\end{frame}


\subsection{Implementation phase}
	\begin{frame}
	\frametitle{Implementation phase}
		\begin{itemize}
			\item After the planning policy makers should transform plans into actions.
			\item We focus on renewable energy policies, in particular the photovoltaic (PV) case.
			\item Many instruments to implement a policy; in the energy case we have, to name a few:
			\begin{itemize}
				\item Feed-in tariffs 
				\item Grants
				\item Fiscal incentives
				\item Low interests/guaranteed loans
				\item Tax exemptions
				\item Green Certificate
			\end{itemize}
		\end{itemize}
	\end{frame}
	
	\begin{frame}
	\frametitle{Implementation phase - Enforcing the policy}
		\begin{itemize}
		\item Which instrument should we use and in which amount to achieve the planned (regional) objective?
			\begin{itemize}
				\item Each instrument has a cost
				\item The plants are installed by citizens and enterprises.
			\end{itemize}
		\item We need to understand the \emph{social reaction} to policy instruments
		\end{itemize}
		\begin{block}<2->{Multi-agent simulation}
		Agent-based economic and social simulation aids the policy maker to evaluate the best implementation strategies 
		\begin{itemize}
			\item Case study: Installed power from PV plants in Emilia-Romagna Region 
		\end{itemize}
		\end{block}
	\end{frame}
		
  	
\section{Simulation Model}
	  	
  	\begin{frame}
	\frametitle{Agent-based model}
	Two types of agents:
		\begin{itemize}
		\item The Region - provides regional incentives on top of national ones to foster the installation of PV panels	
		\item House owners - perform a feasibility study, decide if the investment is profitable from an economic point of view
			\begin{itemize}
			\item Agent parameters: surface of roof, budget, energy consumption, obstinacy..
			\item Global parameters: price of electricity, average cost of PV plants, yearly increase of energy prices, national and regional subsidies..
			\end{itemize}
		\end{itemize}
  	\end{frame}
  	
  	\begin{frame}
	\frametitle{Economic aspects are not enough..}
	 \begin{columns}
		\column{.6\textwidth}
		\begin{figure}[hbt]
			\subfigure[KW of installed PV power in Emilia-Romagna]{
            \label{fig:kw_ER}
            \includegraphics[width=0.7\textwidth]{kw_ER} 
            }\\
			\subfigure[Italian national feed-in tariffs prices, in \euro/kWh]{
            \label{fig:national_inc}
            \includegraphics[width=0.7\textwidth]{national_inc} 
            }
	  	\end{figure}
	  	\column{.4\textwidth}
	  		  	If the decision to install a PV panel was a purely economic choice, one would expect that the better incentive tariffs the more power installed
	  	\begin{block}{}
	  	Non-economic aspects must be considered to understand the relationship between incentives given and the installed PV power
	  	\end{block}
	  \end{columns}
  	\end{frame}
  	
  	\begin{frame}
  	\frametitle{Agents decision algorithm}
  		\begin{columns}
  			\column{.6\textwidth}
		  	\begin{figure}[hbt]
		  	\centering
					\includegraphics[scale=0.22]{decisionAlgorithm}
					\label{decisionAlgorithm}
		  	\end{figure}
		  	\column{.4\textwidth}
		  	\begin{itemize}
		  	\item Feasibility study - physical/financial constraints
		  	\item ROE estimation
		  	\item Social iteration - neighbours behaviour
		  	\item Knowledge diffusion - not every one who would install a panel knows about PV tech possibilities
		  	\end{itemize}
		\end{columns}
  	\end{frame}
  	
\subsection{Parameters influence}
	\begin{frame}
	\frametitle{Parameters influence}
		\begin{itemize}
			\item We wanted to recognize which are the factors with the greatest impact on the agents
			\item We ran several simulations varying the independent parameters in their valid ranges one at a time and keeping fixed the remaining ones
		\end{itemize}
		\begin{block}<2->{}
		The parameters with greater influence on the results are those related to the social aspect of the simulation, together with the knowledge diffusion and the minimum ROE expected; the parameters related to PV technologies shown smaller effect
		\end{block}
	\end{frame}
	
	\begin{frame}
	\frametitle{Social interaction radius}
		\begin{columns}
  			\column{.6\textwidth}
			\begin{figure}
			\includegraphics[scale=0.4]{radius_influence}
			\end{figure}
			\column{.4\textwidth}
			\begin{itemize}
				\item An agent chooses to install a PV panel also depending on the behaviour of his neighbours
				\item The neighbourhood is defined as a circle of the simulation world
			\end{itemize}
		\end{columns}
	\end{frame}
	
	\begin{frame}
	\frametitle{PV panel price decrease}
		\begin{columns}
  			\column{.6\textwidth}
			\begin{figure}
			\includegraphics[scale=0.4]{pv_price_redux_influence}
			\end{figure}
			\column{.4\textwidth}
			\begin{itemize}
				\item The PV panels costs decrease due to technological advancements
				\item Expressed as a percentage
			\end{itemize}
		\end{columns}
	\end{frame}
  	
  	\begin{frame}
	\frametitle{Starting knowledge diffusion}
		\begin{columns}
  			\column{.7\textwidth}
			\begin{figure}
			\includegraphics[scale=0.4]{know_diff_lin_2coeff_start_influence}
			\end{figure}
			\column{.3\textwidth}
			\begin{itemize}
				\item The knowledge diffusion represents the percentage of agents who are aware of PV technology
				\item Defined by the initial percentage and the yearly increase
			\end{itemize}
		\end{columns}
	\end{frame}
	
\subsection{Parameters Tuning}
	\begin{frame}
	\frametitle{Parameters Fine Tuning}
		\begin{itemize}
			\item After evaluating the parameters with greater influence, we want to fine tune them considering the regional data from the past
			\item The purpose is to obtain a simulator able to forecast future trends
			\item We are currently using a tool called \emph{irace}\footnotemark to obtain an automatic optimal configuration for the most important parameters
		\end{itemize}
		\footnotetext[1]{Manuel López-Ibáñez, Jérémie Dubois-Lacoste, Thomas Stützle, and Mauro Birattari. The irace package, Iterated Race for Automatic Algorithm Configuration. Technical Report TR/IRIDIA/2011-004, IRIDIA, Université libre de Bruxelles, Belgium, 2011}
	\end{frame}
	
	\begin{frame}
	\frametitle{Parameters Fine Tuning - First results}
		\begin{columns}
  			\column{.3\textwidth}
  			\begin{itemize}
  				\item The simulated trend follows the kW installed real data
  				\item There's still a lot of work to do..
  			\end{itemize}
  			\column{.7\textwidth}
  			\begin{figure}
			\includegraphics[scale=0.4]{optSim_vs_realData}
			\end{figure}
  		\end{columns}
	\end{frame}
 	
\section{Conclusion and future works}
	
	\begin{frame}
	\frametitle{Conclusion}
	Future research:
	\begin{itemize}
		\item Simulation extensions
		\begin{itemize}
			\item Combination of incentive instruments
			\item More complex social interaction
			\item Agents decisions based on forecasts about PV tech
		\end{itemize}
		\item Simulation validation
		\begin{itemize}
			\item Validating the simulator on real data
			\item Scalability of results
		\end{itemize}
		\item Parameters fine-tuning
		\begin{itemize}
			\item Fine tune the simulation parameters with automatized mechanisms
		\end{itemize}
	\end{itemize}
	\end{frame}
	
	\begin{frame}
      \begin{center}
        Thank You!\\
        ndr.borghesi@gmail.com\\
        \bigskip \bigskip \bigskip \bigskip 
        \includegraphics[scale=.4]{epolicyLowerLogo}
      \end{center}
       
    \end{frame}
\end{document}
